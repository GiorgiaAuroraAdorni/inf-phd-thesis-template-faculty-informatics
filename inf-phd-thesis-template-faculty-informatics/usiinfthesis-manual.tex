%File: manual.tex
%Part of: USI dissertation style
%Author: Jochen Wuttke
%Date: 2008-01-25


\documentclass{article}
\usepackage{booktabs}
\usepackage{url,xspace,array}
\newcommand{\clsusiinf}{\textsf{usiinfthesis}\xspace}
\author{ Domenico Bianculli \and Jochen Wuttke}
\title{\clsusiinf Dissertation Style
  Documentation\footnote{\textsf{usiinf} and USI-INF
  stand for \emph{Universit\`a della Svizzera Italiana (University of
    Lugano) - Faculty of Informatics}.}}

\begin{document}
\maketitle

\section{Overview}

The \clsusiinf\LaTeX\ class for USI-INF dissertations is based on the standard
\textsf{book} class, and provides additional commands, environments
and default text required for theses submitted at USI-INF.
Some parts of the document structure are mandatory, and where
possible, the class file generates those parts automatically. In place
where this is not possible, the sample file and this documentation
provide a guide how to get things right.

In the interest of having a somewhat consistent look-and-feel to all
theses produced at USI-INF, the class file and the commands
provided take care of formatting and document structure as much as
possible. You are not supposed to change any of these commands or
environments. 

\section{Document Structure}

The minimal structure of a dissertation document is shown below:
\begin{verbatim}

\documentclass[]{usiinfthesis}

\title{My Dissertation - A very long title which runs \\ over two lines} 
\author{Philo S. Doctor}
\advisor{The Student's Advisor} 
\Day{21} 
\Month{September} 
\Year{2009} 
\place{Lugano} 
\programDirector{The PhD program Director} 

\committee{%
  \committeeMember{Alonzo Church}{University of California, Los Angeles, USA}
  \committeeMember{Alan M. Turing}{Princeton University, USA}
  %there can as many members as you like
} 

\begin{document}
\maketitle 

\frontmatter 

\begin{abstract}
This is a very abstract abstract.
\end{abstract}

\begin{acknowledgements}

\end{acknowledgements}

\tableofcontents 

\mainmatter

\chapter{Introduction}

\backmatter

\bibliographystyle{alpha}
\bibliography{biblio}

\end{document}
\end{verbatim}

The commands in the preamble should be self-explanatory and are
required to complete the title page and the list of your program
committee members and such like. A good part of the document structure
is encoded in the \verb|\frontmatter|, \verb|\mainmatter|, and
\verb|\backmatter| commands. These commands \emph{must not} be
omitted in your document.

\subsection{The Frontmatter}

The frontmatter of your document contains everything up to and
including the table of contents, and possible lists of figures and
tables. For USI-INF dissertations the format and content of most parts of
the frontmatter is rather rigidly prescribed. The dissertation class
provides commands for all these elements, and most of the commands are
processed internally by the \verb|\frontmatter| command, so you don't
have to worry about those. \textsf{Abstract} and \textsf{Acknowledgements} appear in the
final document in the order you place them in the source. So take care
to write the abstract before the acknowledgements.
Table~\ref{tab:frontmatter} lists all elements that may appear in the
frontmatter, which commands create them, and whether or not they must
be present in your thesis.

\begin{table}
\centering
\begin{tabular}{lp{6cm}l}
\toprule
Content part & created by & mandatory \\
\midrule
Titlepage & \verb|\maketitle| & yes \\
Submission page & \verb|\frontmatter| & yes \\
Declaration of own work & \verb|\frontmatter| & yes \\
Dedication & \verb|\dedication| & no \\
Epigraph & \verb|\openepigraph| & no \\
Abstract & \verb|\begin{abstract} \end{abstract}| & yes \\
Acknowledgements & \verb|\begin{acknowledgements}| \verb|\end{acknowledgements}| & ? \\
Table of Contents & \verb|\tableofcontents| & yes \\
List of Figures & \verb|\listoffigures| & no \\
List of Tables & \verb|\listoftables| & no \\
any other list & ? & no \\
\bottomrule
\end{tabular}
\caption{Parts of the frontmatter}\label{tab:frontmatter}
\end{table}

Extra sections you might want or need in your thesis are a second abstract
in a language different from English (e.g., the language required by
your double-doctorate program), or additional lists of special
elements in the text.
If you want for example a list of algorithms,
you need to use an extra package to produce it. You can include it as
the last part of the frontmatter.
For the extra abstract, you can reuse the \texttt{abstract}
environment with an extra parameter giving the heading of the section
(see reference documentation and example).

By default the lists of figures and tables will be included in the
table of contents if they are present in your document. Most (default)
styles do \emph{not} do that automatically, so if you add a list and
it should appear in the contents, you will have to play with
\texttt{\char`\\addcontentsline} to get what you want.

\subsection{The Main Matter}

The mainmatter of you thesis is the actual content you are supposed to
write. The main matter \emph{must} begin with the command
\verb|\mainmatter| before the first of your own chapters.

All structuring commands (from \verb|\chapter| down to
\verb|\subsection|) can take an optional parameter with a shorter
version of the section title. This shorter title will be used in the 
page header and the table of contents.

\subsection{Appendices}

If you want to have an appendix, use the command \verb|\appendix| to
switch the formatting to ``appendix mode'' and then use the normal
\verb|\chapter| command for each appendix you have.

\subsection{The Backmatter}

The backmatter contains your references, possibly a glossary and an
index, and must be started with the command \verb|\backmatter|.

The references and index (if present) will be included in the table of
contents automatically. An index should start on a recto, that is a
page with an odd page number. The easiest way to achieve this is to
precede the \texttt{\char`\\theindex} command with a
\texttt{\char`\\cleardoublepage}.

The \clsusiinf class does not provide any
command or environment to produce a glossary or any other possible
section that you might want to put in the back of your thesis. As
usual, chapter opened with \texttt{\char`\\chapter} will also appear
in the table of contents.  while chapters opened with
\texttt{\char`\\chapter*} will be ignored for the table of
contents. In both cases, backmatter chapters are \emph{not} numbered.

\section{Command reference}

\reversemarginpar

This command reference lists only commands provided by the USI-INF
dissertation class. Commands inherited from the \texttt{book} class or
any of the loaded packages are not documented here, unless we made
significant changes to them.

\subsection{Preamble Commands and Environments}


\noindent
The research advisor, or main advisor\marginpar{\texttt{\char`\\ advisor}}
responsible for the student submitting the thesis. This command is
mandatory and takes one argument. 

\medskip

\noindent
 The author of this thesis, i.e., you. This\marginpar{\texttt{\char`\\author}} 
command is mandatory and takes one argument.

\medskip

\noindent
 The co-advisor of the student. This\marginpar{\texttt{\char`\\coadvisor}} 
command is optional and should only be used if you have an official
co-advisor. This command takes one argument.

\medskip

\noindent
This environment constructs the list of all your committee members.
\marginpar{\texttt{committee}} You should only use the command
\verb|\committeeMember| inside this environment. This environment is
mandatory and appears on the submission page.

\medskip

Creates an entry for one
committee\marginpar{\texttt{\char`\\committeeMember}}
member in the list of your
committee. This commands takes the name of the member as the first
argument, and her/his affiliation as the second. 

\medskip

\noindent
These commands form the date when the dissertation was
accepted.\marginpar{\texttt{\char`\\Day|Month| Year}} These are
mandatory and appear on the title page and the submission page.

\bigskip

\noindent Inserts an extra page with a
dedication\marginpar{\texttt{\char`\\dedication}} into the
frontmatter. This command is optional and takes the full text of your
dedication as argument.

\medskip


\noindent
Inserts an opening epigraph into the
frontmatter\marginpar{\texttt{\char`\\openepigraph}}. This command is
a variant of the normal \verb|\epigraph| command. It takes the
epigraphs text as the first, and the source as the second parameter.
This command is optional.

\medskip

\noindent
This should be \emph{Lugano}, unless you finish your
\marginpar{\texttt{\char`\\place}}thesis some place
else. This is mandatory and appears in your declaration of own work.

\medskip

The director of the USI-INF PhD
program.\marginpar{\texttt{\char`\\programDirector}} This is mandatory
and appears on the submission page.

\subsection{Text Body Commands and Environments}

Inserts the abstract into the frontmatter.
\marginpar{\texttt{abstract}} Use this environment to
produce an abstract that conforms to the thesis requirements. If you
need a second abstract in another language, use the syntax
\texttt{\char`\\begin\{abstract\}[Sommario]}.

\medskip

Inserts the acknowledgements into the frontmatter.
\marginpar{\texttt{acknowledgements}} Use this environment to produce
an acknowledgements section that conforms to the thesis requirements.

\medskip

\noindent
This command switches formatting\marginpar{\texttt{\char`\\appendix}} and pagination from the main text
body to the form required for appendices. This command is optional and
should only be used if you have appendices. In that case, it must
appear \emph{after} the last chapter of the main text, and
\emph{before} the first appendix.

\medskip

\noindent
This command switches formatting\marginpar{\texttt{\char`\\backmatter}} and pagination to the form used for the
backmatter. This command \emph{must} appear after the last
chapter/appendix and before the (optional) glossary and references.

\medskip

\noindent
Creates most of the frontmatter
pages\marginpar{\texttt{\char`\\frontmatter}} and initialises formatting and
pagination settings. This command \emph{must} appear after
\verb|\maketitle| and before any other commands or text.

\medskip

\noindent
This command must appear after
all\marginpar{\texttt{\char`\\mainmatter}} frontmatter parts and before the
first chapter of the main text body.

\subsection{Class Options}

The default layout produced by the class is targeted to ``electronic''
\marginpar{\texttt{print}} publishing and uses margins consistent with the normal \LaTeX\
\texttt{oneside} option. This option
 switches the layout and various other things to something that is
more suitable for two-sided printing and binding.
 Standard \LaTeX\ options \texttt{oneside} or \texttt{twoside} are disabled.

\medskip

\noindent
By default, the class loads the \texttt{hyperref} package with the proper
options \marginpar{\texttt{nohyper}}. Since the \texttt{hyperref} package redefines
many \LaTeX\ commands, it may conflict with other packages you
use. This option let you  disable the loading of the package.


\section{Restrictions and Requirements}

Commands and document elements listed here may not be changed or used in producing your
thesis.

\section{Required Packages}

The \clsusiinf class makes extensive use of a wide range of
``standard''\footnote{Standard packages are the ones available in a
  modern \LaTeX\ distribution, like \TeX Live
  (\url{http://www.tug.org/texlive}) and Mac\TeX\ (\url{http://www.tug.org/mactex}).}
\LaTeX\ packages. Table~\ref{tab:packages} lists all packages (and options)
that are loaded by the class, and thus do not need to be loaded in
your thesis document.

\begin{table}
\centering
\begin{tabular}{>{\sffamily}l>{\ttfamily}l}
\toprule
Package & Options\\
\midrule
amsmath & \\
book (class) & a4paper, 12pt, onecolumn, final, openright, titlepage
\\
beramono & scaled \\
booktabs & \\
caption & font=sf, labelsep=period \\
epigraph & \\
fancyhdr & \\
fontenc & T1\\
geometry & a4paper \\
graphicx & \\
hyperref &  unicode, plainpages=false, pdfpagelabels, breaklinks\\
hypcap & all \\
mathdesign & charter\\
natbib & square \\
sectsty & \\
textcomp & \\
url & \\

\bottomrule
\end{tabular}
\caption{Required packages and selected options}\label{tab:packages}
\end{table}

The packages \texttt{beramono} and \texttt{mathdesign} select, respectively, the
monospaced and the math fonts.
The class file also uses the \emph{Optima} font package for the sans
serif fonts\footnote{The \emph{Optima} (aka \emph{URW Classico}) font
  is not bundled with \TeX live-based distributions. However, they
  provide a script, \texttt{getnonfreefonts}, for installing extra
  fonts. To install \emph{Optima}, just type \texttt{getnonfreefonts
    classico} on the command line; the script requires \texttt{wget}
  to be installed.}. \texttt{fontenc} and \texttt{textcomp} are required to
make these fonts work properly. 

% If you want to change, you have several easy options:
% \begin{enumerate}
% \item Use a complete fontpackage that loads correct font settings for
% all font types. \texttt{bera} is a good example for such packages,
% even though the font is not that great.
% \item Replace a specific font style by using a package for that. Our
% use of \texttt{beramono} gives you an example for that.
% \item Comment out the three lines of code that load the fonts in the
% class file. They are preceded by the comment \texttt{font settings}.
% This will revert back to \LaTeX's default Computer Modern fonts.
% \end{enumerate}

% Options 1 and 2 do not require you to touch the class file and are
% recommended. Option three, which doesn't give you much benefit anyway,
% is possible, but not recommended.

% It is \emph{not} recommended (actually, in the interest of somehow
% consistent styles in the thesis, it is not \emph{wanted}) that you
% change the font type of structural elements. This means, you can
% change which serif and sans serif fonts you want to use, but you
% should not change \emph{where} serif or sans serif is used.


\section{Complete Document}

The listing below shows the complete structure of your thesis with all
optional content enabled. It is important that you do \emph{not}
change the order of the commands, parts, and sections in your own
thesis.

\newpage %temporary widow

\begin{verbatim}
\documentclass[]{usiinfthesis}

\title{The Title of my Dissertation} %compulsory
\subtitle{Subtitle: Reinventing the World} %optional 
\author{Philo S. Doctor} %compulsory
\advisor{The Student's Advisor} %compulsory
\coadvisor{Co-Advisor} %optional
\Day{Yesterday} %defaults to \today
\Month{September} %compuslory
\Year{2009} %compulsory, put only the year
\place{Lugano} %compulsory
\programDirector{The PhD program Director} %compulsory

\committee{%
  \committeeMember{Alonzo Church}{University of California, Los Angeles, USA}
  \committeeMember{Alan M. Turing}{Princeton University, USA}
  %there can as many members as you like
} %the committee is compulsory

\dedication{To my beloved} %optional
\openepigraph{Someone said \dots}{Someone} %optional

\makeindex %optional, also use \theindex at the end

\begin{document}
\maketitle %generates the titlepage, this is FIXED
\frontmatter %generates the frontmatter, this is FIXED

\begin{abstract}
This is a very abstract abstract.
\end{abstract}

\begin{abstract}[Zusammenfassung]
%creates a new abstract section with "Zusammenfassung" as heading
\end{abstract}

\begin{acknowledgements}
\end{acknowledgements}

\tableofcontents 
\listoffigures %optional
\listoftables %optional
%add any other lists here

\mainmatter

\chapter{Introduction}

\chapter{A chapter title which will run over two lines --- it's for
  testing purpose}

\section{The first section}

\section{The second, math section}

\section[third]{A very very long section, titled ``The third section'', with
  a rather  short text alternative (third)}

\appendix %optional, use only if you have an appendix

\chapter{Some retarded material}
\section{It's over\dots}

\backmatter

\chapter{Glossary} %optional

%\bibliographystyle{alpha} %any style compatible with the natbib package
\bibliographystyle{dcu}
%\bibliographystyle{plainnat}

\bibliography{biblio}

\cleardoublepage %the index starts on a recto!
\theindex %optional, use only if you have an index, must use
          %\makeindex in the preamble

\end{document}
\end{verbatim}

\section{Version History}
\begin{description}
\item[2009/09/30 v.~1.0.8] \ \\  Fixed spelling of the university name.
\item[2009/07/05 v.~1.0.7] \ \\  Fixed header capitalization problems
  in frontmatter. Fixed page number issue with toc, lof, lot
\item[2009/05/05 v.~1.0.6] \ \\ Fixed compatibility issue with
  \textsf{TikZ} package. Date commands have a new format.
\item[2009/04/02 v.~1.0.5 (with fixes)] \ \\ Typos in templated text fixed.
\item[2009/03/27 v.~1.0.5] \ \\ Changes to support SmallCaps in Sans Serif Fonts. 
\item[2009/03/17 v.~1.0.4] \ \\ Cosmetic changes to templated text.
\item[2008/07/25 v.~1.0.3] \ \\ Added support for MSc theses.
\item[2008/07/24 v.~1.0.2] \ \\ Fixed documentation bug.
\item[2008/03/18 v.~1.0.1] \ \\Fixed titleplage bug. \\Fixed ToC, LoF, LoT,
  Bibliography and Index header formatting issues.
\item[2008/03/17 v.~1.0] initial release
\end{description}

\end{document}
% vim:wrap:wm=8:bs=2:expandtab:ts=4:tw=70:
